\documentclass{article}

\usepackage{times}

% Little macro to make keywords obvious.
\newcommand{\kw}[1]{{\tt \mbox{#1}}}

\title{MNI .obj file format}
\author{Robert D. Vincent \\
McConnell Brain Imaging Centre \\
Montreal Neurological Institute \\
Montreal, QC H3A 2B4 Canada \\~\\
}
\date{\today}
\begin{document}
\maketitle
% \newpage
%\tableofcontents
%\newpage

\section{Introduction}

The MNI .tag file format was designed as a simple storage format for
representations of selected points on either a single 3-dimensional
data volumes, or on a pair of such volumes. The .tag file is an ASCII
file format only - it provides no support for other character sets or
binary data.

All of the original design and implementation of this format was the
work of David MacDonald.  This document borrows extensively from David's
existing comments and documentation.

\section{Format}

The file is an ASCII file, and normally uses a single linefeed as a
newline character. Spaces, tabs, and newlines may be used to delimit
fields. Carriage return characters are ignored.

\subsection{Header}
A .tag file always begins with a single line containing the exact text:

\begin{verbatim}
MNI Tag Point File
\end{verbatim}

This header line is mandatory and should appear exactly as shown
(capitalization is significant).

The header is followed by the keyword {\tt Volumes} followed by an
equals sign, either the digit '1' or '2', and a semicolon, e.g. either:
\begin{verbatim}
Volumes = 1;
\end{verbatim}
or:
\begin{verbatim}
Volumes = 2;
\end{verbatim}

This field indicates whether the tag point file applies to one or two
volumes. Again, capitalization is significant.

The next field defines the points themselves. It starts with a keyword:
\begin{verbatim}
Points =
\end{verbatim}
followed by a list of point records. The list of point records is
terminated with another semicolon.

\subsection{Point records}
Each point record simply consists of either 3 or 6 floating-point
numbers, represented in ASCII, and separated by spaces. The floating
point numbers give the X, Y, and Z coordinates in the ``world space'' of
the volume(s) to which the tag applies.

The coordinates of the point are mandatory. Other fields may be included
as needed, as follows:
\begin{itemize}
\item A floating-point weight\footnote{This corresponds to the size when markers are saved as tags in MNI-Display}.
\item An integer structure ID\footnote{This corresponds to a paint label value
  when labels are saved as tags in MNI-Display}.
\item An integer patient ID.
\item A text label.
\end{itemize}

These fields must either {\em all} be present for a tag, they may {\em
 all} be absent, or the text label along may be present. The text label
may be surrounded by quotes to allow it to include space characters.

Each point record normally ends with an end-of-line character, but this
is optional.

\subsection{Comments}
Comments can occur anywhere in the file. They are indicated by the
character '\#' or '\%', and are terminated by an end-of-line character.

There is an informal convention of using comments to indicate the
specific volume files to which the tags apply.
\end{document}

